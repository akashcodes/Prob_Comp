\documentclass[12pt]{article}
\usepackage[ansinew]{inputenc} % ASCII (Western CP)
\usepackage{graphicx}
\usepackage{color}
\usepackage[colorlinks]{hyperref}
\usepackage{geometry}
\usepackage{amsmath}
\geometry{left=2.5cm,right=2.5cm,top=2.5cm,bottom=2.5cm}

\title{Probability and Computing - HW6}
\author{Pang Liang\\ Student No. 201418013229033}

\begin{document}
\maketitle

\section{Problem1}
Given an n-vertex undirected graph $G = (V, E)$, consider the following method of generating an independent set. Given a permutation $\sigma$ of the vertices, define a subset $S(\sigma)$ of the vertices as follows: for each vertex $i, i \in S(\sigma)$ if and only if no neighbor $j$ of $i$ precedes $i$ in the permutation $\sigma$.

\begin{itemize}
\item Show that each $S(\sigma)$ is an independent set in $G$.
\item Suggest a natural randomized algorithm to produce $\sigma$ for which you can show that the expected cardinality of $S(\sigma)$ is $\sum_{i=1}^{n} \frac{1}{d_i+1}$, where $d_i$ denotes the degree of vertex $i$.
\item De-randomizing the above algorithm.
\end{itemize}

Solution:\\
\begin{itemize}
\item
According to the algorithm describe above, we denote the permutation $\sigma$ as $\sigma = {v_1, v_2, \dots, v_i, \dots, v_n}$ where $n$ is the number of vertex. And $S(\sigma)={d_1, d_2, \dots, d_k}$, where $k$ is the size of $S(\sigma)$. Obviously, $d_1 = v_1 \in S(\sigma)$ as the first element. Let's look at the $d_i$ which belongs to $S(\sigma)$. There's no connection with the $d_j$ where $j<i$, because no neighbor of it precedes before. And any $d_j$ where $j>i$ has no connection with $d_i$ for the same reason. So the vertices in $S(\sigma)$ is an independence set in $G$.

\item
With the natural randomized algorithm to produce $\sigma$. The expectation of vertex $v_i$ in the subset $S(\sigma)$ is $E(v_i) = \frac{1}{d_i+1}$, for the reason of that $v_i$ need occur first within its $d_i$ neighbors and itself, so the probability of that is $\frac{1}{d_i+1}$.\\
So consider all of $n$ vertices $E(S(\sigma)) = \sum_{i=1}^{n} \frac{1}{d_i+1}$.

\item
Define 3 kind of vertices set: Unused, S(\sigma), Used. First we random select a vertex in Unused set. Then add this vertex's neighbors and itself into Used set, add itself into $S(\sigma)$, and delete this vertex's neighbors and itself in Unused set. Repeat this step, until the Unused set is empty. 

\end{itemize}

\section{Problem2}
Prove that, for every integer $n$, there exists a way to 2-color the edges of $K_x$ so that there is no monochromatic clique of size $k$ when $x = n - ({}_k^n) 2^{1-({}_2^k)}$. Note that $K_x$ stands for the x-vertex complete graph. (Hint, start by 2-coloring the edges of $K_n$ and fix things up.)

Solution:\\

\section{Problem3}
Prove the following claims.
\begin{itemize}
\item For every integer $n$, there exists a coloring of the edges of the complete graph $K_n$ by two colors so that the total number of monochromatic copies of $K_4$ is at most $({}_4^n) 2^{-5}$.
\item Give a randomized algorithm for finding a coloring with at most $({}_4^n) 2^{-5}$ monochromatic copies of $K_4$ that runs in expected time polynomial in $n$.
\item Show how to construct such a coloring deterministically in polynomial time using the method of conditional expectations.
\end{itemize}

Solution:\\

\section{Problem4}
Do Bernoulli experiment for 20 trials, using a new 1-Yuan coin. Record the result in a
string $s_1s_2 \cdots s_i \cdots s_{20}$, where $s_i$ is 1 if the $i^{th}$ trial gets Head, and otherwise is 0.

1110110100 1100001011

\end{document}
