\documentclass[12pt]{article}
\usepackage[ansinew]{inputenc} % ASCII (Western CP)
\usepackage{graphicx}
\usepackage{color}
\usepackage[colorlinks]{hyperref}
\usepackage{geometry}
\usepackage{amsmath}
\usepackage{amsfonts}
\geometry{left=2.5cm,right=2.5cm,top=2.5cm,bottom=2.5cm}

\title{Probability and Computing - HW8}
\author{Pang Liang\\ Student No. 201418013229033}

\begin{document}
\maketitle

\section{Problem1}
Assume that two states $i$ and $j$ of a Markov chain belong to the same communicating class. Prove that $i$ is recurrent if so is $j$. [Hint: you can prove either straightforwardly or by showing that $i$ is recurrent if and only if $\sum_n p_{ii}^{(n)} = +\infty$. This formula follows from the fact that $p_{ii}^{(n)} = \sum_{n=1}^n r_{ii}^{(m)} p_{ii}^{(n-m)} $, where $r_{ii}^{(m)}$ is the probability that the chain first returns to $i$ at step $m$ if it starts at state $i$].

Solution:\\


\section{Problem2}
Assume that two states $i$ and $j$ of a Markov chain belong to the same communicating class and are recurrent. It is known that $i$ is null recurrent if and only if the multi-step transition probability $p_{ii}^{(n)} $ satisfies $\lim_{n \to +\infty}p_{ii}^{(n)} = 0 $. Prove that $j$ is null recurrent if so is $i$.

Solution:\\

\section{Problem3}
Given a finite markov chain, where finiteness means that there are a finite number of states, prove that
\begin{enumerate}
\item At least one state is recurrent.
\item All recurrent states are positive recurrent.
\end{enumerate}

Solution:\\

\section{Problem4}
Do Bernoulli experiment for 20 trials, using a new 1-Yuan coin. Record the result in a
string $s_1s_2 \cdots s_i \cdots s_{20}$, where $s_i$ is 1 if the $i^{th}$ trial gets Head, and otherwise is 0.

1100111101 0000101001

\end{document}
