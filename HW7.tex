\documentclass[12pt]{article}
\usepackage[ansinew]{inputenc} % ASCII (Western CP)
\usepackage{graphicx}
\usepackage{color}
\usepackage[colorlinks]{hyperref}
\usepackage{geometry}
\usepackage{amsmath}
\usepackage{amsfonts}
\geometry{left=2.5cm,right=2.5cm,top=2.5cm,bottom=2.5cm}

\title{Probability and Computing - HW7}
\author{Pang Liang\\ Student No. 201418013229033}

\begin{document}
\maketitle

\section{Problem1}
Consider a graph in $G_{n,p}$ with $p = c\frac{\ln{n}}{n}$. Use the second moment method to prove that if $c < 1$ then, for any constant $\epsilon > 0$ and for $n$ sufficiently large, the graph has isolated vertices with probability at least $1-\epsilon$.

Solution:\\

We consider the event $X_i$ denotes that the $i^{th}$ vertex is isolated. So
\begin{equation}
    X_i = \left \{ \begin{array}{ll}
    1 & if\ v_i\ is\ isolated \\
    0 & otherwise
    \end{array}
    \right.
\end{equation}
Let
\begin{equation}
    X = \sum_{i=1}^{n} (1-p)^{n-1}.
\end{equation}
so that
\begin{equation}
    E[X] = n (1-p)^{n-1}
\end{equation}
In order to prove that if $c < 1$ then, for any constant $\epsilon > 0$ and for $n$ sufficiently large, the graph has no isolated vertex with probability at most $\epsilon$. That means $Pr(X=0)=o(1)$.\\
We wish to compute
\begin{equation}
    Var[x] = Var[\sum_{i=1}^{n} X_i].
\end{equation}
Applying Lemma 6.9, we see that we need to consider the covariance of the $X_i$.
\begin{equation}
    \begin{split}
    Cov[X_iX_j] &= E[X_iX_j]-E[X_i]E[X_j]\\
    &= (1-p)^{2n-3} - (1-p)^{n-1}*(1-p)^{n-1}\\
    &= p(1-p)^{2n-3}
    \end{split}
\end{equation}
So
\begin{equation}
    Var[X] \le E[X] + \sum Cov[X_iX_j] = E[X] + o(pn^2(1-p)^{2n-3})
\end{equation}
Then
\begin{equation}
    \begin{split}
    Pr(X=0) &\le \frac{Var[X]}{E[X]^2}\\
    &=\frac{1}{n (1-p)^{n-1}} + \frac{p}{1-p}
    \end{split}
\end{equation}
for $p = c\frac{\ln{n}}{n}$ and $c<1$ with $n \to \infty$, $Pr(X=0) \to o(1)$. So the graph has isolated vertices with probability at least $1-\epsilon$.

\section{Problem2}
Prove the Asymmetric Lovasz Local Lemma: Let $\mathbb{A} = \{A_1, \dots, A_n\}$ be a set of finite events over a probability space, and for each $1 \le i \le n$, $\tau(A_i) \in \mathbb{A}$ is such that $A_i$ is mutually independent of all events not in $\tau(A_i)$. If $\sum_{A_j \in \tau(A_i)} Pr(A_j) \le 1/4 $ for all $i$, then $ Pr(\bigwedge_{i=1}^n \bar{A_i}) \ge \prod_{i=1}^n (1 - 2Pr(A_i))> 0$. [Hint: let $x(A_i) = 2Pr(A_i)$ and use the general Lovasz Local Lemma.]

Solution:\\
First we need to prove a lemma that if $0\le a_i \le 1/2$ for all $i=1,2,\dots,n$, then $\prod_{i=1}^n (1-2a_i) \ge 1-2\sum_{i=1}^n a_i$.\\
Induction for $n$. When $n=1$, the inequality holds obviously. Assume that when $n=k$, the inequality holds. Consider the case when $n=k+1$,
\begin{equation}
    \begin{split}
    \prod_{i=1}^{k+1}(1-2a_i) &= \prod_{i=1}^k (1-2a_i) (1-2a_{k+1}) \\
    &\ge (1-2\sum_{i=1}^k a_i)(1-2a_{k+1})\\
    &= 1-2\sum_{i=1}^{k+1} a_i + 4\sum_{i=1}^k a_i a_{k+1}\\
    &\ge 1-2\sum_{i=1}^{k+1} a_i
    \end{split}
\end{equation}
So the inequality holds.

Using the general Lovasz Local Lemma, we set $x(A_i) = 2Pr(A_i)$. Then
\begin{equation}
    \begin{split}
    x(A_i)\prod_{A_j\in \Gamma(A_i)}(1-x(A_j)) &= 2Pr(A_i)\prod_{A_j\in \Gamma(A_i)}(1-2Pr(A_j))\\
    &\ge 2Pr(A_i) (1-2\sum_{A_j\in \Gamma(A_i)}Pr(A_j)\\
    &\ge 2Pr(A_i) (1 - 2 * 1/4)\\
    &= Pr(A_i)
    \end{split}
\end{equation}
So the general Lovasz Local Lemma condition holds. Then we have the result
\begin{equation}
    \begin{split}
    Pr(\bigwedge_{i=1}^n \bar{A_i}) &\ge \prod_{i=1}^n (1 - x(A_i))\\
    &\ge \prod_{i=1}^n (1 - 2Pr(A_i))\\
    &> 0.
    \end{split}
\end{equation}

\section{Problem3}
Given $\beta > 0$, a vertex-coloring of a graph $G$ is said to be $\beta$-frugal if (i) each pair of adjacent vertices has different colors, and (ii) no vertex has $\beta$ neighbors that have the same color.\\
Prove that if $G$ has maximum degree $\Delta \ge \beta^\beta$ with $\beta \ge 2$, then $G$ has a $\beta$-frugal coloring with $16\Delta^{1+1/\beta}$ colors. [Hint: you may want to define two types of events corresponding to the two conditions of being $\beta$-frugal. Then the result in question 1 can be used.]

\section{Problem4}
Let $G = (V,E)$ be an undirected graph and suppose each $v \in V$ is associated with a set $S(v)$ of $8r$ colors, where $r \ge 1$. Suppose, in addition, that for each $v \in V$ and $c \in S(v)$ there are at most $r$ neighbors $u$ of $v$ such that $c$ lies in $S(u)$. \\
Prove that there is a proper coloring of $G$ assigning to each vertex $v$ a color from its class $S(v)$ such that, for any edge $(u, v) \in E$, the colors assigned to $u$ and $v$ are different. You may want to let $A_{u,v,c}$ be the event that $u$ and $v$ are both colored with color $c$ and then consider the family of such events.

\section{Problem5}
Do Bernoulli experiment for 20 trials, using a new 1-Yuan coin. Record the result in a
string $s_1s_2 \cdots s_i \cdots s_{20}$, where $s_i$ is 1 if the $i^{th}$ trial gets Head, and otherwise is 0.

1100111101 0000101001

\end{document}
